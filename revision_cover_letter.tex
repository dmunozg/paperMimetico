\documentclass{letter}
\usepackage{siunitx}
\usepackage[letterpaper,lmargin=3cm,rmargin=3cm]{geometry}

\signature{Diego Muñoz-Gacitúa}

\begin{document}

\begin{letter}{Diego Muñoz-Gacitúa\\ Facultad de ciencias \\ Universidad de
    Chile \\ Las Palmeras \#3425, Ñuñoa, Chile}
  
  \opening{Dear Prof. Schroer}
  
  We are greateful for your consideration of our manuscript titled
  ``Characterization and validation of a membrane mimetic with magnetic
  orienting capabilities and high phospholipid content'' to be published in The
  Journal of Molecular Liquids as a full article.
  
  We would also like to extend our thanks to both of the reviewers, who have
  taken their time to give such a deep and extensive review of our work.
  
  Regarding the observations made by both of the reviewers, we have made multiple changes to
  our manuscript in order to amend the issues they found, and they are listed as
  follows:
  
  \begin{itemize}
  \item Reviewer 1 said: \textit{Fatty acids alone are not important biological membranes components such as sugars and proteins, they are rather part of glycerol and sphingolipids!}\\
    We Agree, fatty acids are not present by themselves in the membrane, but
    rather as a fragment of bigger lipids such as glycolipids, shpingolipids and
    phospholipids among others. And thus, we have changed the phrasing from
    \textit{``...mainly phospholipids and containing fatty acids, sugars,
      cholesterol and proteins, among others.''} to \textit{``mainly phospholipids among other lipids and
      also containing sugars, cholesterol and proteins, among others.''}
    
  \item Reviewer 1 said: \textit{Ref. 25 (Martin et al., 2014) did not show that benzocaine is located around the inner interface of liposomes”, as stated by the authors. Such MD simulations showed that benzocaine gets inside the phospholipid bilayers, but they could not show differences between inner and outer monolayers.}\\
    We have not made ourselves clear in this regard, and we are sorry for that. When we wrote ``inner
    interface'' we meant the area inside the bilayer that is closer to the
    interface. And in that regard, the results by Martin \textit{et al.} show
    that Benzocaine is indeed located in this area at equilibrium. We modified
    all mentions of ``inner interface'' to reflect what we actually meant.
    
  \item Reviewer 1 said: \textit{Phospholipids are important in the arrangement of biological membranes not only because “of specific interactions with certain (arginine) aminoacids”, but because they stabilize the lamellar phase, because they are zwitterionic or anionic (but never cationic to avoid interaction with anionic macromolecules such as nucleic acids) … Please rephrase that.}\\
  We agree, and we have modified such sections to include the role of
  phospholipids in the phase stability\
 
\item Reviewer 1 said: \textit{There is no mention on the size of the bicelle arrangments obtained, to sustain their anisotropy when exposed to the magnetic fields (in the time scale of NMR). What are the range of sizes of the bicelles? Would it be possible to image the bicelles with Atomic Force Microscopy, Nanotracking analysis or….?}\\
  Agreed, and we have added the following sentence to fulfill the missing
  information: \textit{Regarding the size of the aggregate, cryo-TEM imagery of similar
  bicelles (made with TTA instead of SDS)
  results in aggregates of about
  \SI{2}{nm}-\SI{5}{nm}, and we expect for the size of
  the membrane mimetic aggregate to be in the same order of magnitude.}

\item Reviewer 1 said: \textit{Item 3.4.1 Once more the authors states that
    benzocaine inserts the bilayer ang get ``close to the inner interface'' as if
    it was the internal monolayer (see question 2). I think they meant ``inside the
    bilayer, nearby the glycerol moiety / polar head group region''. }\\
  We agree, and as stated before, we have amended all mentions of this ``inner
  interface'' so there would be no misunderstantings.
  
\item Reviewer 1 said: \textit{Fig. 9, what is the reason for the change in
    frequency (-100 Hz to + 100 Hz) observed with levoDOPA, in the acyl chain
    carbons 4-5?}\\
	In this case the effect on the quadrupolar splitting is small and within margin of error, which is why we concluded that the effects of this aminoacid on the bilayer are minor to negligible.

\item Reviewer 1 said: \textit{By the same reasoning, the orientation of bicelles reflect their size / anisotropy in the NMR time scale. It is not that “it orient ifself” as stated here: “Also, the membrane mimetic must orient itself in presence of an external magnetic field. This is so the deuterated probes in the mimetic can produce a quadrupole splitting in a solution 2H-RMN spectrum”. Please rephrase that.}
We agree, the preferred orientation of the bicelle is the result of the anisotropy of its diamagnetic susceptibility. The sentence has been modified to reflect that.

\item Reviewer 1 said \textit{What was the reason to use 8 mg benzocaine and 7 mg levoDOPA? If the authors took the molecular weight of the drugs into consideration, why did they choose to express concentrations in g and not as molar ratios (regarding the bicelle’s components)?}\\
Bicelle components were expressed in weight percentages to ease an eventual future reproduction of the mesophase.
\item Reviewer 1 said: \textit{It would be interesting to include information on the partition coefficient of both drugs.}\\
  Agreed. They have been included in the revised manuscript.
  
  \item Reviewer 1 said: \textit{What about the calculated order parameters? If they were calculated, why not to compare them with the literature?}. Reviewer 2 had a similar concern, and said: \textit{A discussion involving the comparison of the order parameters of the present model with the order parameters from natural phospholipidic membranes is necessary.}\\
  We agree. A comparison between quadrupolar splittings (and hence order parameters) of our proposed membrane mimetic and a DHPC/DMPC bicelle with added cholesterol and polyinsaturated phospholipids has been included, and it reads: "\textit{When comparing the NMR spectrum of this bicelle (figure 4-A) with one taken from a DHPC/DMPC bicelle with cholesterol and polyinstaturated phospholipids (figure 4-B), one can see that DMPC-d$_{54}$ splittings (and hence order parameters) are similar to the splittings in SDS-d$_{25}$, therefore one can conclude that the orientational dynamics of this SDS-containing mimetic are similar to a mimetic made from purely phospholipid}"
  
	\item Regarding the language issues pointed by Reviewer 1, we have revisioned the manuscript and the Highlights file, and fixed a number of typographic errors.
 \end{itemize}

   Finally, we would like to thank you for taking your time to revise our
   manuscript, even through these difficult times.

   \closing{Sincerely,}

 \end{letter}
 \end{document}