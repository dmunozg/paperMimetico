\documentclass{letter}

\signature{Diego Muñoz-Gacitúa}

\begin{document}

\begin{letter}{Diego Muñoz-Gacitúa\\ Facultad de ciencias \\ Universidad de
    Chile \\ Las Palmeras \#3425, Ñuñoa, Chile}
  
  \opening{Dear Prof. Schroer}
  
  We are greateful for your consideration of our manuscript titled
  ``Characterization and validation of a membrane mimetic with magnetic
  orienting capabilities and high phospholipid content'' to be published in The
  Journal of Molecular Liquids as a full article.
  
  We would also like to extend our thanks to both of the reviewers, who have
  taken their time to give such a deep and extensive review of our work.
  
  Regarding the observations both of the reviewers, we have made multiple changes to
  our manuscript in order to amend the issues found by the reviewers, as
  follows:
  
  \begin{itemize}
  \item Reviewer 1 said: \textit{...}\\
    We Agree, fatty acids are not present by themselves in the membrane, but
    rather as a fragment of bigger lipids such as glycolipids, shpingolipids and
    phospholipids among others. And thus, we have changed the phrasing from
    \textit{``...mainly phospholipids and containing fatty acids, sugars,
      cholesterol and proteins, among others.''} to \textit{``mainly phospholipids among other lipids and
      also containing sugars, cholesterol and proteins, among others.''}
    
  \item Reviewer 1 said: \textit{...}\\
    We have not made ourselves clear in this regard. When we wrote ``inner
    interface'' we meant the area inside the bilayer that is closer to the
    interface. And in that regard, the results by Martin \textit{et al.} show
    that Benzocaine is indeed located in this area at equilibrium. We modified
    all mentions of ``inner interface'' to reflect what we actually meant.
    
  \item Reviewer 1 said: \textit{tamaño de la bicela}\\
    
  \item Reviewer 1 said: \textit{...}}\\
  We agree, and we have modified such sections to include the role of
  phospholipids in the phase stability\
 
\item Reviewer 1 said: \textit{...}\\
  Agreed, and we have added the following sentence to fulfill the missing
  information: ``\textit{Regarding the size of the aggregate, cryo-TEM imagery of similar
  bicelles (made with TTA instead of SDS)
  results in aggregates of about
  \SI{2}{nm}-\SI{5}{nm}, and we expect for the size of
  the membrane mimetic aggregate to be in the same order of magnitude.}''
\item Reviewer 1 said: \textit{By the same reasoning, the orientation of
    bicelles reflect their size/anisotropy in the NMR time scale. It is not
    that ``it orient ifself'' as stated here: “Also, the membrane mimetic
    must orient itself in presence of an external magnetic field. This is so the
  deuterated probes in the mimetic can produce a quadrupole splitting in a
  solution 2H-RMN spectrum”. Please rephrase that.}

\item Reviewer 1 said: \textit{Item 3.4.1 Once more the authors states that
    benzocaine inserts the bilayer ang get ``close to the inner interface'' as if
    it was the internal monolayer (see question 2). I think they meant ``inside the
    bilayer, nearby the glycerol moiety / polar head group region''. }\\
  We agree, and as stated before, we have amended all mentions of the ``inner
  interface'' so there would be no misunderstantings.
\item Reviewer 1 said: \textit{Fig. 9, what is the reason for the change in
    frequency (-100 Hz to + 100 Hz) observed with levoDOPA, in the acyl chain
    carbons 4-5?}
  
  

 \end{itemize}

   Finally, we would like to thank you for taking your time to revise our
   manuscript, even through these difficult times.

   \closing{Sincerely,}

 \end{letter}
 \end{document}